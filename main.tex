\documentclass[margin,line]{res}

\oddsidemargin -.5in
\evensidemargin -.5in
\textwidth=6.0in
\itemsep=0in
\parsep=0in
% if using pdflatex:
%\setlength{\pdfpagewidth}{\paperwidth}
%\setlength{\pdfpageheight}{\paperheight} 

\newenvironment{list1}{
  \begin{list}{\ding{113}}{%
      \setlength{\itemsep}{0in}
      \setlength{\parsep}{0in} \setlength{\parskip}{0in}
      \setlength{\topsep}{0in} \setlength{\partopsep}{0in} 
      \setlength{\leftmargin}{0.17in}}}{\end{list}}
\newenvironment{list2}{
  \begin{list}{$\bullet$}{%
      \setlength{\itemsep}{0in}
      \setlength{\parsep}{0in} \setlength{\parskip}{0in}
      \setlength{\topsep}{0in} \setlength{\partopsep}{0in} 
      \setlength{\leftmargin}{0.2in}}}{\end{list}}

\usepackage{hyperref}
\usepackage{verbatim}
\usepackage{enumerate}
% \usepackage{bibentry}
% \usepackage{biblatex}
% \addbibresource{list.bib}

\begin{document}

% \vspace*{-0.8in}
\pagenumbering{gobble}
\name{Jincheng Yang \vspace*{.1in}}

\begin{resume}


\section{\sc Contact Information}
%\vspace{.05in}
\begin{tabular}{@{}p{2.7in}p{3.05in}}        
Department of Mathematics & {\it Voice:}    (+1) 512-317-7231 \\
College of Natural Science & {\it E-mail:} \href{mailto:jcyang@math.utexas.edu}{\sf jcyang@math.utexas.edu}\\
The University of Texas at Austin & {\it LinkedIn:} \href{https://www.linkedin.com/in/jincheng-yang/}{\sf linkedin.com/in/jincheng-yang/}\\
Austin, TX 78712 USA & {\it Website:} \href{https://www.ma.utexas.edu/users/jcyang/}{\sf https://www.ma.utexas.edu/users/jcyang/}\\
\end{tabular}


\section{\sc Research Interests}
Analysis, dynamical systems, partial differential equations, and their application in fluid mechanics
\bigskip

\section{\sc Education}

{\bf The University of Texas at Austin}, Austin, Texas USA\\
\vspace*{-.1in}
\begin{list1}
\item[] Ph.D. Candidate in Mathematics (Pure) \hfill Aug. 2017 - Present
\end{list1}

{\bf Xi'an Jiaotong University} (XJTU), Xi'an, Shaanxi China\\
\vspace*{-.1in}
\begin{list1}
\item[] B.Sc. in Mathematics and Applied Mathematics (Elite Class)\hfill Aug. 2013 - July 2017
\item[] Thesis: \textit{Linear Inviscid Damping of a Shear Flow in a Half Space and in a Finite Channel}
\item[] Advisor: Dongsheng Li and Zhiwu Lin
%\item[] Overall GPA: 94.7/100\ \ \ \ Major GPA: 96.0/100
\end{list1}

% \section{\sc Visiting}

% {\bf Georgia Institute of Technology}, Atlanta, Georgia USA\\
% \vspace*{-.1in}
% \begin{list1}
% \item[] Non-degree\hfill Jan. 2016 - May 2016
% \item[] School of Mathematics and Language Institute Visiting Honors Student Program
% %\item[] Language GPA: 4.33/4.33\ \ \ \  Major GPA: 4.0/4.33
% \end{list1}

% {\bf Columbia University in the City of New York}, New York City, New York USA\\
% \vspace*{-.1in}
% \begin{list1}
% \item[] Non-degree\hfill Jan. 2015 - May 2015
% \item[] Visiting Undergraduate Student
% %\item[] Overall GPA: 4.165/4.33\ \ \ \ Major GPA: 4.165/4.33
% \end{list1}


\section{\sc Honors and Awards} 
%{\bf National Second Prize in CUMCM} (Contemporary Undergraduate Mathematical Contest in Modeling) \hfill {\bf 2014}\\
{\bf First Everest Research Scholarship}, %awarded by 
XJTU \hfill {\bf 2014}\\
%{\bf Microsoft Scholarship for Young Scholar}, only 36 students in China are awarded by Microsoft Research - Asia \hfill {\bf 2015}\\
{\bf National Scholarship}, 
%top 1\% students, awarded by 
Ministry of Education, China \hfill {\bf 2016}\\
{\bf Pacemaker to Outstanding Student}, %highest honor for students at 
XJTU \hfill {\bf 2016}\\
{\bf Frank Gerth III Graduate Excellence Award}, UT Austin \hfill {\bf 2018}\\
{\bf Senate of College Council's TA of the Year}, UT Austin \hfill {\bf 2019}\\
{\bf Frank Gerth III Teaching Excellence Award}, UT Austin \hfill {\bf 2020}

\begin{comment}
\section{\sc Standardized Tests}
\begin{list2}
\item GRE General: 322 (V152, Q170, W3.5) \hfill { Dec. 6, 2015}
\item GRE Subject (Mathematics): 910 (99\% Below) \hfill { Apr. 16, 2016}
\item TOEFL iBT: 109 (R29, L29, S24, W27) \hfill { Apr. 22, 2016}
\end{list2}
\end{comment}

\section{\sc Academic Experience}

% {\bf Xi'an Jiaotong University}, Xi'an, Shaanxi China

% \vspace{-.3cm}
% {\em Programmer/Writer} \hfill {\bf May 2013 - Sept. 2014}\\
% Programming and editing a college-level textbook which applies RAPTOR language in algorithm courses for non-computer science major. I wrote and checked coding for all examples and exercises.

%{\em Research Trainee} \hfill {\bf July 2015 - Sept. 2015}\\
%Includes reading and discussing {\it Singular Integrals and Differentiability Properties of Functions} by Elias M. Stein led by Dongsheng Li.

% {\em Teaching Assistant} \hfill {\bf June 2016}\\
% Duties includes administrative responsibilities and oral translation for a course taught by Dan G. Sykes from Pennsylvania State University.

%{\bf Columbia University in the City of New York}, New York City, New York USA

%\vspace{-.3cm}
%{\em Research Trainee} \hfill {\bf Jan. 2015 - May 2015}\\
%Includes reading and discussing {\it Parital Differential Equations} by Lawrance C. Evans led by Ovidiu Savin.

{\bf Georgia Institute of Technology}, Atlanta, Georgia USA

% \vspace{-.3cm}
% {\em Researcher} \hfill {\bf Jan. 2016 - Oct. 2016}\\
% Research includes studying the effect of density variation of fluids on the linear inviscid damping of Couette flow with Zhiwu Lin.

% \vspace{-.3cm}
% {\em Visiting Student} \hfill {\bf Feb. 2016 - May 2016}\\
% Research includes studying with Zhiwu Lin. 
%By means of Fourier decomposition and solving frequency equations with hypergeometric functions, w
% We showed the decay rate for velocity and density variation to linearized Euler equations near stratified Couette flow under optimal regularity. We found the sharp decay rate depends solely on Richardson number. 

% \vspace{-.3cm}
{\em Visiting Research Student} \hfill {\bf Feb. 2017 - May 2017}\\
Research includes the effect of density variation of fluids on the inviscid damping of stratified Couette flow, and the barotropic instability of shear flows for incompressible fluids with Coriolis effects.
% For a class of shear flows, we developed a new method to find the sharp stability conditions. We studied the flow with sinus profile in details and obtained the sharp stability boundary in the whole parameter space, which corrects previous results in the fluid literature.

{\bf The University of Texas at Austin}, Austin, Texas USA

% \vspace{-.3cm}
{\em Teaching Assistant} \hfill {\bf Sept. 2017 - Present}\\
Teaching assistant for differential/integral/vector calculus, differential equations and linear algebra.


\section{\sc Publications}

\begin{enumerate}[\hspace{-.13in}1.]
    \item Xie, T., Cheng, X. \& Yang, J. (2014) RAPTOR Program Designing Tutorial. Beijing: Tsinghua University Press.
    \item Yang, J. \& Lin, Z. (2018) \textit{Linear Inviscid Damping for Couette Flow in Stratified Fluid}, Journal of Mathematical Fluid Mechanics, \textbf{20}: 445-472. \href{https://doi.org/10.1007/s00021-017-0328-3}{https://doi.org/10.1007/s00021-017-0328-3}
    \item Lin, Z., Yang, J. \& Zhu, H. (2020) \textit{Barotropic Instability of Shear Flows}, Studies in Applied Mathematics, \textbf{144}: 289-326. 
    \href{https://doi.org/10.1111/sapm.12297}{https://doi.org/10.1111/sapm.12297} 
\end{enumerate}





% \begin{enumerate}
% \item \bibentry{YangLin2018}
% \item \bibentry{LinYangZhu2020}
% \end{enumerate}

\section{\sc Preprint}

\begin{enumerate}[\hspace{-.13in}1.]
    \item Yang, J. (2020) \textit{Construction of Maximal Functions associated with Skewed Cylinders Generated by Incompressible Flows and Applications}, submitted, \href{https://arxiv.org/abs/2008.05588}{arXiv:2008.05588}
    \item Vasseur, A. \& Yang, J. (2020) \textit{Second Derivatives Estimate of Suitable Solutions to the 3D Navier-Stokes Equations}, submitted, \href{https://arxiv.org/abs/2009.14291}{arXiv:2009.14291}
\end{enumerate}



% \begin{comment}
% \section{\sc Computer Skills} 
% \begin{list2}
% \item Math-related:  Matlab, Mathematica, \LaTeX
% \item Programming:  Java, C++, Processing, RAPTOR
% \item Other: Audio and Video editing, Website and Publication Layout Design.
% \end{list2}
% \end{comment}


\end{resume}
\end{document}